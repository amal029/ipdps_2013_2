\documentclass[10pt]{article}

\begin{document}

\section{Abstract}
\label{sec:abstract-2}

\section{Introduction and problem description}
\label{sec:intr-probl-descr}

This section should motivate the problem that we are facing or will face
with advancement of computing technology. We need to motivate the
solution of using CE and SA. Why both? Mention that the problem is even
for homogeneous case NP-hard.

In this paper we might be able to even do tiling and finer grained
allocation using split par actors. We might be able to mention something
about the polyhedral model and how we are better than them. I can
produce graphs with separate par actors and see how they perform with
CE.

\section{Background}
\label{sec:background}

This should give the theoretical background of the whole process. For
example, describe the application graphs formally. Introduce the
mathematical notation behind the cost functions and the graph
theory. Introduce what we mean by throughput (formally).

We need to give the background about CE and SA (all the math or no math
behind it).

\section{Modelling heterogeneity of the application graphs and the
  execution platforms}
\label{sec:model-heter-appl}

Describe the kind of heterogeneity that exists in the application
graph. Nodes requiring vector instructions. Nodes that are stores. How
the communication is modelled, etc. How we do it in the CE and SA
methods. Look at other people and refer them (describe how we differ
compared to them). Maybe the comparison should be put in a different
section altogether.

This is the main part of the paper: Describe how the platform is
modeled. Here as well.

\section{Experimental results}
\label{sec:experimental-results}

We divide this into multiple sub-sections: (1) Give a brief background
of the experimental setup. This should include the description of cross
entropy. (2) The other sections should be partitioned into different
parts within the table. We also need synthetic graphs.

\section{Related work}
\label{sec:related-work}



\section{Conclusions and future work}
\label{sec:concl-future-work}

We have none

\end{document}
