\section{Notations and formalization of the problem statement}
\label{sec:relat-notat-form}

The overall application partitioning problem onto a heterogeneous
execution architecture can formulated in terms of a graph partitioning
problem. Given two graphs, the task-graph representing the application
and the resource-graph representing the underlying topology. How does
one map the task-graph onto the resource-graph.


\subsection{Formalizing the task-graph}
\label{sec:form-task-graph}

A task-graph is a graph weighted directed graph $G_t(V_t,E_t)$, such
that each vertex $V_t$ is an execution node in the application, and
\mbox{$E_t \subseteq (V_t \times V_t)$}, shows the communication edges
between the vertices. Each vertex $V_t$ is decorated with weights:
$w^t_0(i), \forall i \in V_t$ and $w^t_1(i), \forall i \in V_t$. Weight
$w^t_0$ is a function on some vertex $i \in V_t$, that maps the number
of instructions to be carried out at node $i$ to an integer
value. Similarly, $w^t_1$ is a function on some vertex $i \in V_t$ that
maps the vector length that is required by the node $i$ again to an
integer value. Each edge is decorated by a weight $w^e(e), \forall e \in
(V_t \times V_t)$. Weight $w^e$ represents the number of bytes that need
to be transfered from the location of the data-store ($i$) to the
utilization node ($j$).


\subsection{Formalizing the resource graph}
\label{sec:form-reso-graph}

The system resources are represented by a weighted undirected graph
$G_r(V_r,C_r)$. Where $V_r$ represents a processing element in the
underlying resource graph, while the edge $C_r \subseteq (V_r \times
V_r)$, represents the communication links. Each vertex is decorated with
a weights $W^r_0(i)$ and $W^r_1(i), \forall i \in V_r$, which map the
\textit{Million Instructions Per Second} (MIPS) count of the vertex to
an integer value, and the vector capacity of that vertex again to an
integer domain, respectively. Every, communication link is weighted with
the bandwidth capacity denoted by $W^c(c), \forall c \in (V_r \times
V_r)$.

\subsection{Formalizing the objective: the latency of the application}
\label{sec:form-latency-appl}

The total application latency in a parallel setting is the addition of
the computation time and the communication time. Given some application
node $i \in V_t$ mapped to some resource $j \in V_r$. The latency for
that node is computed as: $((w^t_1(i)/W^r_1(j)\times w^t_0(i))/W^r_0(c))
+ (w^e(e)/W^c(c)) | e = (i,k), k \neq i, \forall k \in V_t, c = (j,l), l
\neq j, \forall l \in V_r $. In this formulation for some given
task-graph node $i$, we first calculate the number of vectorized
instructions that need to be performed (by diving the required vector
length with the vector capacity of the resource node), this gives us the
total number of vector instructions that would be performed on the
resource node $j$. Next, we multiply the number of vector instructions
to be performed by the number of iterative (non-vectorized) instructions
required, this in turn gives us the total number of instructions to be
performed by that task-graph node. Finally, we find the computation
latency by divining this total number of instructions with the MIPS
value of the resource vertex. The communication on the other hand is a
quite simple, we calculate the communication latency, by dividing the
number of required bits to be transfered by the bandwidth of the
resource.

Given the task-graph and the resource-graph, let $\zeta$ be all possible
mappings of the application on the resource-graph. For a particular
mapping $\mathcal{M}$ defined as $\zeta_\mathcal{M}$ on some resource $s
\in V_r$ the mapping latency is defined as:

\begin{equation}
  \begin{array}{c}
    Latency^{\zeta_\mathcal{M}}_s = \\
    \\
    \sum_{\forall i \in V_t \wedge
      \zeta_\mathcal{M} = s} ((w^t_1(i)/W^r_1(s)\times w^t_0(i))/W^r_0(c))
    \\
    +
    \\
    \sum_{\forall i \in V_t \wedge
      \zeta_\mathcal{M} = s} w^e(e) / W^c(c)\\ 
    s.t., e = (i,k), k \neq i, \forall k
    \in V_t \wedge\  c = (s,l), l \neq s, \forall l \in V_r
  \end{array}
  \label{eq:1}
\end{equation}

Finally, the complete application latency can then be defined as: 
\begin{equation}
  \label{eq:2}
  Latency^{\zeta_\mathcal{M}} = max_{s}
  ({Latency^{\zeta_\mathcal{M}}_s}), \forall s \in V_t
\end{equation}

The objective of our framework is to minimize the total application
latency as described in Equation~(\ref{eq:2}).

%%% Local Variables: 
%%% mode: latex
%%% TeX-master: "bare_conf"
%%% End: 
