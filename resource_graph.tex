\section{Generating the resource graph}
\label{sec:gener-reso-graph}


\begin{figure}[t!]
  \centering
  % \includegraphics[scale=0.42]{./figures/jacobi2d}
  \scalebox{0.38}{\section{Generating the resource graph}
\label{sec:gener-reso-graph}


\begin{figure*}[t!]
  \centering
  % \includegraphics[scale=0.42]{./figures/jacobi2d}
  \scalebox{0.45}{\section{Generating the resource graph}
\label{sec:gener-reso-graph}


\begin{figure*}[t!]
  \centering
  % \includegraphics[scale=0.42]{./figures/jacobi2d}
  \scalebox{0.45}{\section{Generating the resource graph}
\label{sec:gener-reso-graph}


\begin{figure*}[t!]
  \centering
  % \includegraphics[scale=0.42]{./figures/jacobi2d}
  \scalebox{0.45}{\input{./figures/resource_graph.pdf_t}}
  \caption{An example resource graph}
  \label{fig:1r}
\end{figure*}

We use synthetically generated graphs to test our framework, because in
a heterogeneous setup, we do not know the exact nature of the processing
elements being utilized. For example, a setup might consist of a Tesla
Nvidia GPU core connected to an Atmel micro-processor. It is also
possible that a CELL processing unit is connected to a X86 processing
element. Moreover, we cannot judge beforehand the type of bus
connections one might have on this multi-core system. In order to
overcome these issues we take a more statical view of the process. We
use synthetically generated topologies to test our framework. One such
synthetically generated resource graph is shown in Figure~\ref{fig:1r}.

In the resource graph in Figure~\ref{fig:1r}, the processing elements
are denoted via boxes. Each processing element consists of two
decorations following the previously defined formalism in
Section~\ref{sec:form-reso-graph}. In Figure~\ref{fig:1r} the processing
elements are labeled from \texttt{A} through to \texttt{P} and are
placed in a 2D tiled topology. For processing element \texttt{A} the
MIPS count is provided as $W^A_0$, while the vector length count is
provided as $W^A_1$. The bandwidth connecting the various processing
elements are denoted via $W^c$.

%%% Local Variables: 
%%% mode: latex
%%% TeX-master: "bare_conf"
%%% End: 
}
  \caption{An example resource graph}
  \label{fig:1r}
\end{figure*}

We use synthetically generated graphs to test our framework, because in
a heterogeneous setup, we do not know the exact nature of the processing
elements being utilized. For example, a setup might consist of a Tesla
Nvidia GPU core connected to an Atmel micro-processor. It is also
possible that a CELL processing unit is connected to a X86 processing
element. Moreover, we cannot judge beforehand the type of bus
connections one might have on this multi-core system. In order to
overcome these issues we take a more statical view of the process. We
use synthetically generated topologies to test our framework. One such
synthetically generated resource graph is shown in Figure~\ref{fig:1r}.

In the resource graph in Figure~\ref{fig:1r}, the processing elements
are denoted via boxes. Each processing element consists of two
decorations following the previously defined formalism in
Section~\ref{sec:form-reso-graph}. In Figure~\ref{fig:1r} the processing
elements are labeled from \texttt{A} through to \texttt{P} and are
placed in a 2D tiled topology. For processing element \texttt{A} the
MIPS count is provided as $W^A_0$, while the vector length count is
provided as $W^A_1$. The bandwidth connecting the various processing
elements are denoted via $W^c$.

%%% Local Variables: 
%%% mode: latex
%%% TeX-master: "bare_conf"
%%% End: 
}
  \caption{An example resource graph}
  \label{fig:1r}
\end{figure*}

We use synthetically generated graphs to test our framework, because in
a heterogeneous setup, we do not know the exact nature of the processing
elements being utilized. For example, a setup might consist of a Tesla
Nvidia GPU core connected to an Atmel micro-processor. It is also
possible that a CELL processing unit is connected to a X86 processing
element. Moreover, we cannot judge beforehand the type of bus
connections one might have on this multi-core system. In order to
overcome these issues we take a more statical view of the process. We
use synthetically generated topologies to test our framework. One such
synthetically generated resource graph is shown in Figure~\ref{fig:1r}.

In the resource graph in Figure~\ref{fig:1r}, the processing elements
are denoted via boxes. Each processing element consists of two
decorations following the previously defined formalism in
Section~\ref{sec:form-reso-graph}. In Figure~\ref{fig:1r} the processing
elements are labeled from \texttt{A} through to \texttt{P} and are
placed in a 2D tiled topology. For processing element \texttt{A} the
MIPS count is provided as $W^A_0$, while the vector length count is
provided as $W^A_1$. The bandwidth connecting the various processing
elements are denoted via $W^c$.

%%% Local Variables: 
%%% mode: latex
%%% TeX-master: "bare_conf"
%%% End: 
}
  \caption{An example resource graph}
  \label{fig:1r}
\end{figure}

We use synthetically generated graphs to test our framework, because in
a heterogeneous setup, we do not know the exact nature of the processing
elements being utilized. For example, a setup might consist of a Tesla
Nvidia GPU core connected to an Atmel micro-processor. It is also
possible that a CELL processing unit is connected to a X86 processing
element. Moreover, we cannot judge beforehand the type of bus
connections one might have on this multi-core system. In order to
overcome these issues we take a more statical view of the process. We
use synthetically generated topologies to test our framework. One such
synthetically generated resource graph is shown in Figure~\ref{fig:1r}.

In the resource graph in Figure~\ref{fig:1r}, the processing elements
are denoted via boxes. Each processing element consists of two
decorations following the previously defined formalism in
Section~\ref{sec:form-reso-graph}. In Figure~\ref{fig:1r} the processing
elements are labeled from \texttt{A} through to \texttt{P} and are
placed in a 2D tiled topology. For processing element \texttt{A} the
MIPS count is provided as $W^A_0$, while the vector length count is
provided as $W^A_1$. The bandwidth connecting the various processing
elements are denoted via $W^c$.

%%% Local Variables: 
%%% mode: latex
%%% TeX-master: "bare_conf"
%%% End: 
