\section{Simulated annealing}
\label{sec:simulated-annealing}

Simulated Annealing~\cite{skir83} is a generic probabilistic algorithm
for finding the global optimum of a given function,
$\mathcal{F}:\mathcal{R}\rightarrow\mathcal{R}$, which has a large
search space.  Simulated Annealing in most of the cases is cheaper than
exhaustive enumeration, which is exhaustively listing the entire search
space. Since it is a monte carlo method it will always terminate and the
solution need not be the most optimal but in general considered to be a
good approximation of the optimum.

% \subsection{Overview}

% <<<<<<< HEAD
% In general SA is a non-greedy heuristic algorithm which explores the
% search space by moving from a higher temperature to a lower
% temperature. The algorithm always accepts moves to a better solution,
% determined by evaluating an \emph{objective function} and sometimes
% accpets moves to worse states with a probability that changes with
% temperatures. Initially, when the temperature is high, the algorithm is
% more tolerable and accepts moves to worse states with a higher
% probability. As the temperature falls down, this probability decreases
% and the alogrithm is less tolerant towards solutions with a worse value
% for the objective function.

% This is the reason why SA is not a purely greedy algorithm as it accepts
% moves to bad states with some probability which enables it to move out
% of local maximas\/minimas. But the algorithm becomes more greedier as
% the temperature decreases. Figure~\ref{fig:temperature_drop} shows how
% the temperature drops and how this acceptance probability changes with
% this drop in temperature.

% \subsection{From the mapping problem standpoint}

% \begin{algorithm}
%   \DontPrintSemicolon
%   \KwIn{Initial Mapping $\zeta_0$ and Starting Temperature $T_0$}
%   \KwOut{Best Mapping $\zeta_{best}$}
%     $\zeta \leftarrow \zeta_0$\;
%     $C \leftarrow OBJECTIVE\_FUNCTION(\zeta_0)$\;
%     $\zeta_{best} \leftarrow \zeta$\;
%     $C_{best} \leftarrow C$\;
%     \Return $\zeta_{best}$
%   \caption{$Simulated\_Annealing(\zeta_0, T_0)$}
%   \label{algo:SA}
% \end{algorithm}

% Figure~\ref{algo:SA} represents the pseudocode for our algorithm. We
% begin by choosing some initial starting point for our state space
% exploration, which from the mapping problem standpoint is some initial
% mapping of tasks from $V_t$ to PEs $V_r$. We choose our initial mapping
% to be all tasks from $V_t$ mapped onto some random PE $j \in V_r$. This
% is a good idea because we have observed that this puts tightly coupled
% tasks, i.e. tasks that communicate heavily, on the same PE since moving
% them onto different PEs would incur more costs and would thus lead to a
% bad state.
% =======
In general SA is a non-greedy heuristic algorithm which explores the search
space by moving from a higher temperature to a lower temperature. The algorithm
always accepts moves to a better solution, determined by evaluating an
\textit{objective function} and sometimes accpets moves to worse states with a
probability that changes with temperatures. Initially, when the temperature is
high, the algorithm is more tolerable and accepts moves to worse states with a
higher probability. As the temperature falls down, this probability decreases
and the alogrithm is less tolerant towards solutions with a worse value for the
objective function.

This is the reason why SA is not a purely greedy algorithm as it accepts moves
to bad states with some probability which enables it to move out of local
maximas/minimas. But the algorithm becomes more greedier as the temperature
decreases. Figure~\ref{fig:temperature_drop} shows how the temperature drops and
how this acceptance probability changes with this drop in temperature.

\subsection{Conventional Simulated Annealing : From the mapping problem standpoint}

\begin{scriptsize}
\begin{algorithm}[h!]
\caption{$Simulated\_Annealing(\zeta_0, \mathcal{T}_0)$}
\label{algo:SA}
% \DontPrintSemicolon
\KwIn{Initial Mapping $\zeta_0$ and Starting Temperature $\mathcal{T}_0$}
\KwOut{Best Mapping $\zeta_{best}$}
$\zeta_{current} \leftarrow \zeta_0$ \;
$C_{current} \leftarrow OBJECTIVE\_FUNCTION(\zeta_0)$ \;
$\zeta_{best} \leftarrow \zeta_{current}$\;
$C_{best} \leftarrow C_{current}$\;
$R \leftarrow 0$\;
\For {$i \leftarrow 0\ to\ \infty$}{
  $\mathcal{T}_{current} \leftarrow NEXT\_TEMPERATURE(\mathcal{T_0,i})$\;
  $\zeta_{new} \leftarrow NEXT\_STATE(\zeta_{current}, \mathcal{T})$\;
  $C_{new} \leftarrow OBJECTIVE\_FUNCTION(\zeta_{new})$\;
  $\Delta C \leftarrow C_{new} - C_{current}$\;
  $r \leftarrow RAND()$\;
  $p \leftarrow ACCEPTANCE_PROBABILITY(\Delta C, \mathcal{T}_{current})$\;
  \If{$\Delta C \textless 0$ or $r \textless p$}{
    \If{$C_{new} \textless C_{best}$}{
      $\zeta_{best} \leftarrow \zeta_{new}$\;
      $C_{best} \leftarrow C_{new}$\;
      $\zeta_{current} \leftarrow \zeta_{new}$\;
      $C_{current} \leftarrow C_{new}$\;
      $R \leftarrow 0$\;
    }
  }
  \Else{
    \If{$\mathcal{T}_{current} \leq \mathcal{T}_f$}{
      $R \leftarrow R+1$
      \If{$R \geq R_{max}$}{break}
    }
  }
}
\Return $\zeta_{best}$
\end{algorithm}
\end{scriptsize}

Algorithm~\ref{algo:SA} represents the pseudocode for the algorithm discussed in
~\cite{hors06}. The \textit{OBJECTIVE\_FUNCTION} evaluates execution time of a
specific mapping by calling the scheduler. $\zeta_0, \zeta_{current}$ and
$\mathcal{T}_0, \mathcal{T}_{current}$ are the intial and current mapping and
temperatures respectively. The \textit{NEXT\_STATE} function moves a random task
to a random PE different than the original PE.The \textit{NEXT\_TEMPERATURE} is
chosen so that the annealing schedule is proportional to the number of tasks and
resources and is given by,

\begin{equation}
\textit{NEXT\_TEMPERATURE}(\mathcal{T}_0, i) = \mathcal{T}_0*q^{\lfloor
\frac{i}{L} \rfloor}
\end{equation}

where q is the proprtion of temperature preserved in each temperature level. In
their paper, q was set as $0.95$. L is the number of iterations in one
temperature level. They achieve multiple iterations in one temperature level by
using the \texttt{floor} function, since $\lfloor \frac{i}{L} \rfloor$ moves to
a new value only at every integral multiple of $L$.

\begin{equation}
L = |V_t|*(|V_r| - 1)
\end{equation}

They, also modify the \textit{ACCEPTANCE\_PROBABILITY} from its traditional form
of $\frac{1}{1+exp(\frac{\Delta C}{\mathcal{T}})}$ to a form to adapt
automatically to different cost function value ranges and graphs with different
task execution times. This is done by normalizing the change in cost, $\Delta
C$,

\begin{equation}
ACCEPTANCE\_PROBABILITY(\Delta C, \mathcal{T}) = \frac{1}{1+exp(\frac{\Delta
C}{0.5C_0\mathcal{T}})}
\end{equation}

The initial temperature $\mathcal{T}_0$ of the annealing schedule is selected as
follows,
\begin{equation}
\mathcal{T}_0 = \frac{kt_{max}}{t_{minsum}}
\end{equation}
where $t_{max}$ is the maximum execution time for any task $t_i \in V_t$ on any
PE $r_j \in V_r$ and $t_{minsum}$ is the sum of execution time of all tasks $\in
V_t$ on the fastest PE $r_{fastest} \in V_r$.

The final temperature of the annealing schedule is selected as follows,
\begin{equation}
\mathcal{T}_f = \frac{t_{min}}{kt_{maxsum}}
\end{equation}
where $t_{min}$ is the minimum execution time for any task$t_i \in V_t$ on any
PE $r_j \in V_r$ and $t_{maxsum}$ is the sum of execution time of all tasks $\in
V_t$ on the slowest PE $r_{slowest} \in V_r$. In both the cases, $k$ is a
constant and $k=1$ is suitable for all our experiments.

\subsubsection{Our Improved SA Approach}

We made some changes to the original algorithm and associated heuristics as
prescribed by Heikki et al~\cite{hors06}. The reasons behind these changes are two folds.
\begin{itemize}
\item To suit the heterogeneous nature of the resources
\item To improve the annealing schedule and hence the final solution
\end{itemize}

Some of the changes are as follows,

\begin{itemize}

\item We begin by choosing some initial starting point for our state space exploration, which from
the mapping problem standpoint is some initial mapping of tasks from $V_t$ to
PEs $V_r$, to be all tasks from $V_t$ mapped onto
some random PE $j \in V_r$. This is a good idea because we have observed that
this puts tightly coupled tasks, i.e. tasks that communicate heavily, on the
same PE since moving them onto different PEs would incur more communication
costs and would thus lead to a bad state.

\item The \textit{OBJECTIVE\_FUNCTION} evaluates the latency of execution of
the mapping $\zeta$ on $V_r$ as described earlier in
Section~\ref{sec:relat-notat-form}. 

\item The \mbox{\textit{NEXT\_STATE($\zeta_{current}$, $T_{current}$)}} function
describes how we are going to move to a new state. This is one of the most
important contributions of this paper. We make this movement to a new state a
function of the current temperature $\mathcal{T}_{current}$ as well. Consider a
given mapping $\zeta_\mathcal{M}$ which maps $V_t$ to $V_r$. When the
temperature is high, we allow a lot of elements in this mapping of the form $v_i
\leftarrow v_j$ to change to mappings of the form $v_i \leftarrow v_k$. This
means we allow a lot of tasks to migrate to different PEs when the temperature
is high. As the temperature decreases, we restrict the motion of these tasks,
meaning we allow only fewer tasks to migrate to different PEs and enforce a
condition on the rest of the tasks to stay at the PE they are currently in. It
should also be kept in mind that during every iteration, the tasks that are
allowed to move are selected at random while the number of tasks that have to be
moved will be guided by temperature. The reason for doing this is two fold,
\begin{itemize}
\item We want to explore different corners of the search space when the
temperature is high
\item As the temperature drops, we want to reduce our search space to the
current best solutions neighbourhood
\end{itemize}
This can be thought of as doing a global state space exploration at the beginning
stages of the optimization, i.e. when temperature is high, as we want to get a
sense of which neighbourhood gives a good result. Once we find a good
negihbourhood we start to reduce our radius of search by allowing only limited
number of tasks to move to different PEs. This results in a localized search
when the temperature drops. We found this to be quite effective in terms of
finding effective mappings as our results from~\ref{results-section} shows.

\item The \textit{NEXT\_TEMPERATURE} function, which is defined as follows, is
also changed to prevent multiple mapping iterations in one temperature level. We
want to avoid this as we do not want to spend a majority of the annealing
schedule searching for global optimums. 

\begin{equation}
\textit{NEXT\_TEMPERATURE}(\mathcal{T}_0, i) = \mathcal{T}_0*q^{i}{L}
\end{equation}

Figure~\ref{temperature-drop-graphs} shows how the temperature varies with
iterations. It clearly shows how the temperature drops quicker for smaller
values of $q$ and slower for larger values. Even though we removed the concept
of multiple number of mapping iterations per temperature level from the previous
algorithm, we retained the $q$ parameter as we wanted to control how
quickly/slowly the temperature must drop.

\item We use the value for the initial and final temperature for the annealing
schedule from the paper written by Heikki et al.~\cite{hors06}. Since the
objective of this article is to map heterogeneous applications onto
heterogeneous architectures, these heuristics have to be changed as well. We 
calculate the fastest and slowest procesorss by multiplying each processor's
capabilities ($W^r_0(i) * W^r_1(i)$) and sorting them. 

\item The rest of the parameters remain the same as~\cite{hors06} and are subject to future work
for changes if necessary.

\end{itemize} 

%%% Local Variables: 
%%% mode: latex
%%% TeX-master: "bare_conf"
%%% End: 
