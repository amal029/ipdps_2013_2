\section{Simulated annealing}
\label{sec:simulated-annealing}

Simulated Annealing~\cite{kirkpatrick} is a generic probabilistic algorithm for
finding the global optimum of a given function,
$\mathcal{F}:\mathcal{R}\rightarrow\mathcal{R}$, which has a large search space.
Simulated Annealing in most of the cases is cheaper than exhaustive enumeration,
which is exhaustively listing the entire search space. Since it is a monte carlo
method it will always terminate and the solution need not be the most optimal
but in general considered to be a good approximation of the optimum.

\subsection{Overview}

In general SA is a non-greedy heuristic algorithm which explores the search
space by moving from a higher temperature to a lower temperature. The algorithm
always accepts moves to a better solution, determined by evaluating an
\textit{objective function} and sometimes accpets moves to worse states with a
probability that changes with temperatures. Initially, when the temperature is
high, the algorithm is more tolerable and accepts moves to worse states with a
higher probability. As the temperature falls down, this probability decreases
and the alogrithm is less tolerant towards solutions with a worse value for the
objective function.

This is the reason why SA is not a purely greedy algorithm as it accepts moves
to bad states with some probability which enables it to move out of local
maximas\/minimas. But the algorithm becomes more greedier as the temperature
decreases. Figure~\ref{fig:temperature_drop} shows how the temperature drops and
how this acceptance probability changes with this drop in temperature.

\subsection{From the mapping problem standpoint}

\begin{algorithm}[1]
\caption{$Simulated\_Annealing(\zeta_0, \mathcal{T}_0)$}
\label{algo:SA}
\DontPrintSemicolon
\KwIn{Initial Mapping $\zeta_0$ and Starting Temperature $\mathcal{T}_0$}
\KwOut{Best Mapping $\zeta_{best}$}
$\zeta \leftarrow \zeta_0$ \\
$C \leftarrow OBJECTIVE\_FUNCTION(\zeta_0)$ \\
$\zeta_{best} \leftarrow \zeta$
$C_{best} \leftarrow C$
\end{algorithm}

Algorithm~\ref{algo:SA} represents the pseudocode for our algorithm. We begin by
choosing some initial starting point for our state space exploration, which from
the mapping problem standpoint is some initial mapping of tasks from $V_t$ to
PEs $V_r$. We choose our initial mapping to be all tasks from $V_t$ mapped onto
some random PE $j \in V_r$. This is a good idea because we have observed that
this puts tightly coupled tasks, i.e. tasks that communicate heavily, on the
same PE since moving them onto different PEs would incur more costs and would thus
lead to a bad state.

The \textit{OBJECTIVE\_FUNCTION} evaluates the latency of execution of the
mapping $\zeta$ on $V_r$ as described earlier in
Section~\ref{problem-definition}. $\zeta_0$ is the initial mapping chosen as
discussed in the previous paragraph, $\mathcal{T}_0$ is the starting
temperature of the annealing process and $\zeta_{current}$ and $T_{current}$ are
the current mapping and temperature respectively. $\zeta_{new}$ is the next
mapping as prescribed by the \mbox{\textit{NEXT\_STATE($\zeta_{current}$,
$T_{current}$)}} function and $\mathcal{C}_{new}$ is the cost of the new mapping
as evaluated by \textit{OBJECTIVE\_FUNCTION($\zeta_{new}$)}. $\mathcal{T}$
changes according to the \textit{NEXT\_TEMPERATURE} which is defined in the
coming section.

The \textit{NEXT\_STATE} function, which is one of the contributions of this
paper, is defined in the coming section. \textit{RAND} returns a value between
$[0, 1)$ which is then compared with the value returned by
\textit{ACCEPTANCE\_PROBABILITY}. We choose to accept a worse state based on
these two values.

\subsection{Parameter Selection}
The quality of the final solution entirely depends on the selection of values
for the parameters of interest. This configures the annealing schedule and the
acceptance probability and move functions.

We use the value for the initial and final temperature for the annealing
schedule from the paper written by Heikki et al.~\ref{parameters-SA}. 

The initial temperature $\mathcal{T}_0$ of the annealing schedule is selected as follows,
\begin{equation}
\mathcal{T}_0 = \frac{kt_{max}}{t_{minsum}}
\end{equation}
where $t_{max}$ is the maximum execution time for any task $t_i \in V_t$ on any
PE $r_j \in V_r$ and $t_{minsum}$ is the sum of execution time of all tasks $\in
V_t$ on the fastest PE $r_{fastest} \in V_r$.

The final temperature of the annealing schedule is selected as follows,
\begin{equation}
\mathcal{T}_f = \frac{t_{min}}{kt_{maxsum}}
\end{equation}
where $t_{min}$ is the minimum execution time for any task$t_i \in V_t$ on any
PE $r_j \in V_r$ and $t_{maxsum}$ is the sum of execution time of all tasks $\in
V_t$ on the slowest PE $r_{slowest} \in V_r$. In both the cases, $k$ is a
constant and $k=1$ is suitable for all our experiments.

Since the objective of this article is to map heterogeneous applications onto heterogeneous architectures, these heuristics have to be changed as well. When we  

%%% Local Variables: 
%%% mode: latex
%%% TeX-master: "bare_conf"
%%% End: 
