\section{Simulated annealing}
\label{sec:simulated-annealing}

Simulated Annealing~\cite{kirkpatrick} is a generic probabilistic
algorithm for finding the global optimum of a given function,
$\mathcal{F}:\mathcal{R}\rightarrow\mathcal{R}$, which has a large
search space.  Simulated Annealing in most of the cases is cheaper than
exhaustive enumeration, which is exhaustively listing the entire search
space. Since it is a monte carlo method it will always terminate and the
solution need not be the most optimal but in general considered to be a
good approximation of the optimum.

\subsection{Overview}

In general SA is a non-greedy heuristic algorithm which explores the
search space by moving from a higher temperature to a lower
temperature. The algorithm always accepts moves to a better solution,
determined by evaluating an \emph{objective function} and sometimes
accpets moves to worse states with a probability that changes with
temperatures. Initially, when the temperature is high, the algorithm is
more tolerable and accepts moves to worse states with a higher
probability. As the temperature falls down, this probability decreases
and the alogrithm is less tolerant towards solutions with a worse value
for the objective function.

This is the reason why SA is not a purely greedy algorithm as it accepts
moves to bad states with some probability which enables it to move out
of local maximas\/minimas. But the algorithm becomes more greedier as
the temperature decreases. Figure~\ref{fig:temperature_drop} shows how
the temperature drops and how this acceptance probability changes with
this drop in temperature.

\subsection{From the mapping problem standpoint}

\begin{algorithm}
  \DontPrintSemicolon
  \KwIn{Initial Mapping $\zeta_0$ and Starting Temperature $T_0$}
  \KwOut{Best Mapping $\zeta_{best}$}
    $\zeta \leftarrow \zeta_0$\;
    $C \leftarrow OBJECTIVE\_FUNCTION(\zeta_0)$\;
    $\zeta_{best} \leftarrow \zeta$\;
    $C_{best} \leftarrow C$\;
    \Return $\zeta_{best}$
  \caption{$Simulated\_Annealing(\zeta_0, T_0)$}
  \label{algo:SA}
\end{algorithm}

Figure~\ref{algo:SA} represents the pseudocode for our algorithm. We
begin by choosing some initial starting point for our state space
exploration, which from the mapping problem standpoint is some initial
mapping of tasks from $V_t$ to PEs $V_r$. We choose our initial mapping
to be all tasks from $V_t$ mapped onto some random PE $j \in V_r$. This
is a good idea because we have observed that this puts tightly coupled
tasks, i.e. tasks that communicate heavily, on the same PE since moving
them onto different PEs would incur more costs and would thus lead to a
bad state.

\subsection{Parameter Selection}


%%% Local Variables: 
%%% mode: latex
%%% TeX-master: "bare_conf"
%%% End: 
