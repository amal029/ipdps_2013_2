\section{Experiments and Results}

We show the speedup obtained using our improved heuristics against the
conventional heuristics for Simulated Annealing as prescribed
by~\cite{hors06}. We also compare the results we obtain against the
K-way graph partitioning algorithm~\cite{gkar95} and heterogeneous bin
packing~\cite{mmar11}.

\subsection{K-way graph partitioning}
\label{sec:k-way-graph}

Graph partitioning plays an important role in the multi-processor and
VLIW scheduling and partitioning
algorithms~\cite{aale01,kpur99,enys98}. K-way graph partitioning is an
important algorithm, which partitions a given graph into K or less
parts, resulting in load balanced allocations. K-way partitioning mixed
with min-edge cuts can form a good tool to partition a task-graph onto a
multi-processor system, resulting in equal utilization of PEs and
reduced communication costs. We have utilized the Metis~\cite{gkar95}
graph partitioning tool to perform a K-way graph partitioning onto
heterogeneous multi-processor architectures for comparison
purposes.

Metis is a graph partitioner, which implements K-way partitioning with
min-edge cut as the primary objective. The weights on the graph nodes
are represented as constraints. Each graph node can have multiple-node
weights, representing different criteria. The edges between nodes can be
weighted themselves, but as opposed to nodes, edges can only be
decorated with a single weight. Moreover, the unique point about Metis
is that one can use a concept called ``tp-weights'', which give
weightage, to different node constraints when performing load-balancing
during K-way graph partitioning. These are the only features of Metis
that we required when implementing our algorithm. Metis provides a
number of other features, which the readers can lookup in~\cite{gkar95}.

The gist of our K-way task partitioning approach onto a heterogeneous
multi-core architecture is as follows:

\begin{itemize}

\item Our resource graph is first described as a simple graph in
  metis. In this description each of the two capabilities $W^i_0$ and
  $W^i_1$ are described as two constraints for each node in the
  graph. The communication bandwidth is described as edge weights in
  metis.

\item Once we have the resource graph in the metis format we calculate
  the tp-weights. There are two tp-weights generated, one for each of
  the resource node capabilities. Let us revisit the 4 $\times$ 4 mesh
  shown in Figure~\ref{fig:1r}. For each processor the MIPS tp-weight is
  calculated by the formula: $W^i_0/\sum_{\forall i \in
    V_r}(W^i_0)$. Similarly, we can easily calculate the tp-weight
  metric for each PEs vector length capability as: $W^i_1/\sum_{\forall
    i \in V_r}(W^i_1)$. These fractions give a relative approximation of
  the capabilities of each PE compared to the other.

\item The nodes in our task graph are described as node in the metis
  format. The two requirements ($R^i_0$ and $R^i_1$) for each task node
  are described as two constraints in the metis node format. The edge
  weights in our task graph are described as edge weights in the metis
  graph format.

\item Once we have the task graph described in the metis format along
  with the tp-weight metric for each PE in the resource graph. We ask
  for a $|V_r|$ partition from metis, giving the tp-weight metric for
  each constraint of the task graph.

\item The resultant partition is then used to calculate the objective in
  Equation~(\ref{eq:2}).

\end{itemize}


\subsection{Heterogeneous bin packing}
\label{sec:heter-bin-pack}

Heuristic bin packing solutions have given good results in the general
case~\cite{ecof78}. Comparing with the heterogeneous bin packing
heuristic~\cite{mmar11} allows us to gauge the effectiveness of our
algorithm against a standard technique.

Let $\mathcal{I}$ be the items to be accommodated into the bins and let
$\mathcal{K}$ be the set of bins available.  From the standpoint of the
mapping problem, $\mathcal{I}$ refers to the set of task graph nodes
($|V_t|$) and $\mathcal{K}$ refers to the nodes in the resource graph
($|V_r|$). Similar to the Knapsack problem~\cite{sski08}, by which A-BFD
is inspired, each element $i \in \mathcal{I},\ \mathcal{K}$ has two
constraints on them represented by $c_i$ (cost), which translates to the
PE capability $W^i_0$ and $V_i$ (volume), which translates to the
capability $W^i_1$, respectively.
% Right off the bat, this seems to be a significant advantage, our
% framework has over heterogeneous bin packing, as \textit{A-BFD} only
% works with two constraints whereas our framework poses no such
% restrictions.
\textit{A-BFD} proceeds to sort $\mathcal{I}$ according
to non-increasing order of their volume and sorts $\mathcal{K}$
according to non-increasing order of the ratio $c_i/V_i$. Then, it
proceeds to allocate items from $\mathcal{I}$ into best bins $b \in
\mathcal{S}$. A ``best" bin, i.e., the bin with maximum free space, is
defined as the bin volume minus the sum of volumes of the items loaded
into it. % In the mapping problem setting, this becomes a double edged
% sword as it boils down to a single constraint solving problem giving
% higher priority to the second constraint (vector requirement of the
% application-tasks $w^t_1(i)$ and vector capability of the resources
% $W^r_1(i)$).

The post pass in \textit{A-BFD} chooses every bin that has atleast one
item allocated to it and tries to find an empty bin, that has a higher
or equal volume than the allocated volume on the chosen bin but also has
a lower cost. If it finds such an empty bin, then it transfers all the
items allocated to the chosen bin to the newly found empty bin which is
cheaper. One of the main advantages of \textit{A-BFD} is that it is very
fast with a best case complexity of $O(N_\mathcal{I})$ without the post
pass, where $N_\mathcal{I}$ is the number of items (number of tasks
$|V_t|$ in the application graph $G_t$). Including the post pass, the
best case complexity becomes $O(N_\mathcal{I} + N_\mathcal{K})$ where
$N_\mathcal{K}$ is the number of bins (number of PEs $|V_R|$ in the
resource graph $G_r$).

\subsection{The experimental setup}
\label{sec:experimental-setup}

\subsubsection{The resource graph setup}
\label{sec:resource-graph-setup}


\subsubsection{The task graph setup}
\label{sec:task-graph-setup}


\subsection{The results}
\label{sec:results}


\subsubsection{Comparison with K-way partitioning}
\label{sec:comparison-with-k}

\subsubsection{Comparison with heterogeneous bin packing}
\label{sec:comp-with-heter}


\subsubsection{Comparison with established SA}
\label{sec:comp-with-establ}


\subsubsection{Effect of varying temperature}
\label{sec:varying-temperature}






%%% Local Variables: 
%%% mode: latex
%%% TeX-master: "bare_conf"
%%% End: 
