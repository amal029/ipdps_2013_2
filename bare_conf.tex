% -*- mode:latex; mode:flyspell -*-
%% bare_conf.tex
%% V1.3
%% 2007/01/11
%% by Michael Shell
%% See:
%% http://www.michaelshell.org/
%% for current contact information.
%%
%% This is a skeleton file demonstrating the use of IEEEtran.cls
%% (requires IEEEtran.cls version 1.7 or later) with an IEEE conference paper.
%%
%% Support sites:
%% http://www.michaelshell.org/tex/ieeetran/
%% http://www.ctan.org/tex-archive/macros/latex/contrib/IEEEtran/
%% and
%% http://www.ieee.org/

%%*************************************************************************
%% Legal Notice:
%% This code is offered as-is without any warranty either expressed or
%% implied; without even the implied warranty of MERCHANTABILITY or
%% FITNESS FOR A PARTICULAR PURPOSE! 
%% User assumes all risk.
%% In no event shall IEEE or any contributor to this code be liable for
%% any damages or losses, including, but not limited to, incidental,
%% consequential, or any other damages, resulting from the use or misuse
%% of any information contained here.
%%
%% All comments are the opinions of their respective authors and are not
%% necessarily endorsed by the IEEE.
%%
%% This work is distributed under the LaTeX Project Public License (LPPL)
%% ( http://www.latex-project.org/ ) version 1.3, and may be freely used,
%% distributed and modified. A copy of the LPPL, version 1.3, is included
%% in the base LaTeX documentation of all distributions of LaTeX released
%% 2003/12/01 or later.
%% Retain all contribution notices and credits.
%% ** Modified files should be clearly indicated as such, including  **
%% ** renaming them and changing author support contact information. **
%%
%% File list of work: IEEEtran.cls, IEEEtran_HOWTO.pdf, bare_adv.tex,
%%                    bare_conf.tex, bare_jrnl.tex, bare_jrnl_compsoc.tex
%%*************************************************************************

% *** Authors should verify (and, if needed, correct) their LaTeX system  ***
% *** with the testflow diagnostic prior to trusting their LaTeX platform ***
% *** with production work. IEEE's font choices can trigger bugs that do  ***
% *** not appear when using other class files.                            ***
% The testflow support page is at:
% http://www.michaelshell.org/tex/testflow/



% Note that the a4paper option is mainly intended so that authors in
% countries using A4 can easily print to A4 and see how their papers will
% look in print - the typesetting of the document will not typically be
% affected with changes in paper size (but the bottom and side margins will).
% Use the testflow package mentioned above to verify correct handling of
% both paper sizes by the user's LaTeX system.
%
% Also note that the "draftcls" or "draftclsnofoot", not "draft", option
% should be used if it is desired that the figures are to be displayed in
% draft mode.
%
\documentclass[10pt, conference, compsocconf]{IEEEtran}
% Add the compsocconf option for Computer Society conferences.
%
% If IEEEtran.cls has not been installed into the LaTeX system files,
% manually specify the path to it like:
% \documentclass[conference]{../sty/IEEEtran}





% Some very useful LaTeX packages include:
% (uncomment the ones you want to load)


% *** MISC UTILITY PACKAGES ***
%
%\usepackage{ifpdf}
% Heiko Oberdiek's ifpdf.sty is very useful if you need conditional
% compilation based on whether the output is pdf or dvi.
% usage:
% \ifpdf
%   % pdf code
% \else
%   % dvi code
% \fi
% The latest version of ifpdf.sty can be obtained from:
% http://www.ctan.org/tex-archive/macros/latex/contrib/oberdiek/
% Also, note that IEEEtran.cls V1.7 and later provides a builtin
% \ifCLASSINFOpdf conditional that works the same way.
% When switching from latex to pdflatex and vice-versa, the compiler may
% have to be run twice to clear warning/error messages.






% *** CITATION PACKAGES ***
%
%\usepackage{cite}
% cite.sty was written by Donald Arseneau
% V1.6 and later of IEEEtran pre-defines the format of the cite.sty package
% \cite{} output to follow that of IEEE. Loading the cite package will
% result in citation numbers being automatically sorted and properly
% "compressed/ranged". e.g., [1], [9], [2], [7], [5], [6] without using
% cite.sty will become [1], [2], [5]--[7], [9] using cite.sty. cite.sty's
% \cite will automatically add leading space, if needed. Use cite.sty's
% noadjust option (cite.sty V3.8 and later) if you want to turn this off.
% cite.sty is already installed on most LaTeX systems. Be sure and use
% version 4.0 (2003-05-27) and later if using hyperref.sty. cite.sty does
% not currently provide for hyperlinked citations.
% The latest version can be obtained at:
% http://www.ctan.org/tex-archive/macros/latex/contrib/cite/
% The documentation is contained in the cite.sty file itself.






% *** GRAPHICS RELATED PACKAGES ***
%
\ifCLASSINFOpdf
  \usepackage[pdftex]{graphicx}
  % declare the path(s) where your graphic files are
  % \graphicspath{{../pdf/}{../jpeg/}}
  % and their extensions so you won't have to specify these with
  % every instance of \includegraphics
  % \DeclareGraphicsExtensions{.pdf,.jpeg,.png}
\else
  % or other class option (dvipsone, dvipdf, if not using dvips). graphicx
  % will default to the driver specified in the system graphics.cfg if no
  % driver is specified.
  % \usepackage[dvips]{graphicx}
  % declare the path(s) where your graphic files are
  % \graphicspath{{../eps/}}
  % and their extensions so you won't have to specify these with
  % every instance of \includegraphics
  % \DeclareGraphicsExtensions{.eps}
\fi
% graphicx was written by David Carlisle and Sebastian Rahtz. It is
% required if you want graphics, photos, etc. graphicx.sty is already
% installed on most LaTeX systems. The latest version and documentation can
% be obtained at: 
% http://www.ctan.org/tex-archive/macros/latex/required/graphics/
% Another good source of documentation is "Using Imported Graphics in
% LaTeX2e" by Keith Reckdahl which can be found as epslatex.ps or
% epslatex.pdf at: http://www.ctan.org/tex-archive/info/
%
% latex, and pdflatex in dvi mode, support graphics in encapsulated
% postscript (.eps) format. pdflatex in pdf mode supports graphics
% in .pdf, .jpeg, .png and .mps (metapost) formats. Users should ensure
% that all non-photo figures use a vector format (.eps, .pdf, .mps) and
% not a bitmapped formats (.jpeg, .png). IEEE frowns on bitmapped formats
% which can result in "jaggedy"/blurry rendering of lines and letters as
% well as large increases in file sizes.
%
% You can find documentation about the pdfTeX application at:
% http://www.tug.org/applications/pdftex





% *** MATH PACKAGES ***
%
%\usepackage[cmex10]{amsmath}
% A popular package from the American Mathematical Society that provides
% many useful and powerful commands for dealing with mathematics. If using
% it, be sure to load this package with the cmex10 option to ensure that
% only type 1 fonts will utilized at all point sizes. Without this option,
% it is possible that some math symbols, particularly those within
% footnotes, will be rendered in bitmap form which will result in a
% document that can not be IEEE Xplore compliant!
%
% Also, note that the amsmath package sets \interdisplaylinepenalty to 10000
% thus preventing page breaks from occurring within multiline equations. Use:
%\interdisplaylinepenalty=2500
% after loading amsmath to restore such page breaks as IEEEtran.cls normally
% does. amsmath.sty is already installed on most LaTeX systems. The latest
% version and documentation can be obtained at:
% http://www.ctan.org/tex-archive/macros/latex/required/amslatex/math/





% *** SPECIALIZED LIST PACKAGES ***
%
%\usepackage{algorithmic}
% algorithmic.sty was written by Peter Williams and Rogerio Brito.
% This package provides an algorithmic environment fo describing algorithms.
% You can use the algorithmic environment in-text or within a figure
% environment to provide for a floating algorithm. Do NOT use the algorithm
% floating environment provided by algorithm.sty (by the same authors) or
% algorithm2e.sty (by Christophe Fiorio) as IEEE does not use dedicated
% algorithm float types and packages that provide these will not provide
% correct IEEE style captions. The latest version and documentation of
% algorithmic.sty can be obtained at:
% http://www.ctan.org/tex-archive/macros/latex/contrib/algorithms/
% There is also a support site at:
% http://algorithms.berlios.de/index.html
% Also of interest may be the (relatively newer and more customizable)
% algorithmicx.sty package by Szasz Janos:
% http://www.ctan.org/tex-archive/macros/latex/contrib/algorithmicx/




% *** ALIGNMENT PACKAGES ***
%
%\usepackage{array}
% Frank Mittelbach's and David Carlisle's array.sty patches and improves
% the standard LaTeX2e array and tabular environments to provide better
% appearance and additional user controls. As the default LaTeX2e table
% generation code is lacking to the point of almost being broken with
% respect to the quality of the end results, all users are strongly
% advised to use an enhanced (at the very least that provided by array.sty)
% set of table tools. array.sty is already installed on most systems. The
% latest version and documentation can be obtained at:
% http://www.ctan.org/tex-archive/macros/latex/required/tools/


%\usepackage{mdwmath}
%\usepackage{mdwtab}
% Also highly recommended is Mark Wooding's extremely powerful MDW tools,
% especially mdwmath.sty and mdwtab.sty which are used to format equations
% and tables, respectively. The MDWtools set is already installed on most
% LaTeX systems. The lastest version and documentation is available at:
% http://www.ctan.org/tex-archive/macros/latex/contrib/mdwtools/


% IEEEtran contains the IEEEeqnarray family of commands that can be used to
% generate multiline equations as well as matrices, tables, etc., of high
% quality.


%\usepackage{eqparbox}
% Also of notable interest is Scott Pakin's eqparbox package for creating
% (automatically sized) equal width boxes - aka "natural width parboxes".
% Available at:
% http://www.ctan.org/tex-archive/macros/latex/contrib/eqparbox/





% *** SUBFIGURE PACKAGES ***
\usepackage[tight,footnotesize]{subfigure}
% subfigure.sty was written by Steven Douglas Cochran. This package makes it
% easy to put subfigures in your figures. e.g., "Figure 1a and 1b". For IEEE
% work, it is a good idea to load it with the tight package option to reduce
% the amount of white space around the subfigures. subfigure.sty is already
% installed on most LaTeX systems. The latest version and documentation can
% be obtained at:
% http://www.ctan.org/tex-archive/obsolete/macros/latex/contrib/subfigure/
% subfigure.sty has been superceeded by subfig.sty.



%\usepackage[caption=false]{caption}
%\usepackage[font=footnotesize]{subfig}
% subfig.sty, also written by Steven Douglas Cochran, is the modern
% replacement for subfigure.sty. However, subfig.sty requires and
% automatically loads Axel Sommerfeldt's caption.sty which will override
% IEEEtran.cls handling of captions and this will result in nonIEEE style
% figure/table captions. To prevent this problem, be sure and preload
% caption.sty with its "caption=false" package option. This is will preserve
% IEEEtran.cls handing of captions. Version 1.3 (2005/06/28) and later 
% (recommended due to many improvements over 1.2) of subfig.sty supports
% the caption=false option directly:
%\usepackage[caption=false,font=footnotesize]{subfig}
%
% The latest version and documentation can be obtained at:
% http://www.ctan.org/tex-archive/macros/latex/contrib/subfig/
% The latest version and documentation of caption.sty can be obtained at:
% http://www.ctan.org/tex-archive/macros/latex/contrib/caption/




% *** FLOAT PACKAGES ***
%
%\usepackage{fixltx2e}
% fixltx2e, the successor to the earlier fix2col.sty, was written by
% Frank Mittelbach and David Carlisle. This package corrects a few problems
% in the LaTeX2e kernel, the most notable of which is that in current
% LaTeX2e releases, the ordering of single and double column floats is not
% guaranteed to be preserved. Thus, an unpatched LaTeX2e can allow a
% single column figure to be placed prior to an earlier double column
% figure. The latest version and documentation can be found at:
% http://www.ctan.org/tex-archive/macros/latex/base/



%\usepackage{stfloats}
% stfloats.sty was written by Sigitas Tolusis. This package gives LaTeX2e
% the ability to do double column floats at the bottom of the page as well
% as the top. (e.g., "\begin{figure*}[!b]" is not normally possible in
% LaTeX2e). It also provides a command:
%\fnbelowfloat
% to enable the placement of footnotes below bottom floats (the standard
% LaTeX2e kernel puts them above bottom floats). This is an invasive package
% which rewrites many portions of the LaTeX2e float routines. It may not work
% with other packages that modify the LaTeX2e float routines. The latest
% version and documentation can be obtained at:
% http://www.ctan.org/tex-archive/macros/latex/contrib/sttools/
% Documentation is contained in the stfloats.sty comments as well as in the
% presfull.pdf file. Do not use the stfloats baselinefloat ability as IEEE
% does not allow \baselineskip to stretch. Authors submitting work to the
% IEEE should note that IEEE rarely uses double column equations and
% that authors should try to avoid such use. Do not be tempted to use the
% cuted.sty or midfloat.sty packages (also by Sigitas Tolusis) as IEEE does
% not format its papers in such ways.





% *** PDF, URL AND HYPERLINK PACKAGES ***
%
%\usepackage{url}
% url.sty was written by Donald Arseneau. It provides better support for
% handling and breaking URLs. url.sty is already installed on most LaTeX
% systems. The latest version can be obtained at:
% http://www.ctan.org/tex-archive/macros/latex/contrib/misc/
% Read the url.sty source comments for usage information. Basically,
% \url{my_url_here}.





% *** Do not adjust lengths that control margins, column widths, etc. ***
% *** Do not use packages that alter fonts (such as pslatex).         ***
% There should be no need to do such things with IEEEtran.cls V1.6 and later.
% (Unless specifically asked to do so by the journal or conference you plan
% to submit to, of course. )


% correct bad hyphenation here
\hyphenation{op-tical net-works semi-conduc-tor}


\begin{document}
%
% paper title
% can use linebreaks \\ within to get better formatting as desired
\title{An improved simulated annealing heuristic for partitioning
  heterogeneous applications onto heterogeneous architectures}


% author names and affiliations
% use a multiple column layout for up to two different
% affiliations

\author{\IEEEauthorblockN{Authors Name/s per 1st Affiliation (Author)}
\IEEEauthorblockA{line 1 (of Affiliation): dept. name of organization\\
line 2: name of organization, acronyms acceptable\\
line 3: City, Country\\
line 4: Email: name@xyz.com}
\and
\IEEEauthorblockN{Authors Name/s per 2nd Affiliation (Author)}
\IEEEauthorblockA{line 1 (of Affiliation): dept. name of organization\\
line 2: name of organization, acronyms acceptable\\
line 3: City, Country\\
line 4: Email: name@xyz.com}
}

% conference papers do not typically use \thanks and this command
% is locked out in conference mode. If really needed, such as for
% the acknowledgment of grants, issue a \IEEEoverridecommandlockouts
% after \documentclass

% for over three affiliations, or if they all won't fit within the width
% of the page, use this alternative format:
% 
%\author{\IEEEauthorblockN{Michael Shell\IEEEauthorrefmark{1},
%Homer Simpson\IEEEauthorrefmark{2},
%James Kirk\IEEEauthorrefmark{3}, 
%Montgomery Scott\IEEEauthorrefmark{3} and
%Eldon Tyrell\IEEEauthorrefmark{4}}
%\IEEEauthorblockA{\IEEEauthorrefmark{1}School of Electrical and Computer Engineering\\
%Georgia Institute of Technology,
%Atlanta, Georgia 30332--0250\\ Email: see http://www.michaelshell.org/contact.html}
%\IEEEauthorblockA{\IEEEauthorrefmark{2}Twentieth Century Fox, Springfield, USA\\
%Email: homer@thesimpsons.com}
%\IEEEauthorblockA{\IEEEauthorrefmark{3}Starfleet Academy, San Francisco, California 96678-2391\\
%Telephone: (800) 555--1212, Fax: (888) 555--1212}
%\IEEEauthorblockA{\IEEEauthorrefmark{4}Tyrell Inc., 123 Replicant Street, Los Angeles, California 90210--4321}}




% use for special paper notices
%\IEEEspecialpapernotice{(Invited Paper)}




% make the title area
\maketitle


\begin{abstract}

  In this paper we present a simulated annealing based partitioning
  technique for heterogeneous applications onto heterogeneous processing
  architectures. A heterogeneous application is one where tasks differ
  in the amount of computation and communication that they perform. A
  heterogeneous architecture on the other hand, is a composition of
  processing elements that differ in their capabilities to perform
  computation and communication. Such applications are very common and
  heterogeneous architectures are becoming more common in today's
  environment. Task partitioning onto homogeneous architectures in-order
  to extract parallelism, is a known NP-hard problem. Heterogeneity
  exponentially complicates the aforementioned partitioning problem. A
  number of heuristic approaches have been proposed for heterogeneous
  partitioning some using simulated annealing, to overcome the NP-hard
  nature of the problem. The novelty of our approach is two fold: (1) we
  go a step further than the currently proposed scientific literature,
  considering heterogeneity at levels of: task parallelism, data
  parallelism, and communication. (2) We quantitatively show an
  improvement in the simulated annealing algorithm, both in terms of
  runtime and resulting application latencies, provided the neighbors in
  the simulated annealing strategy are guided with the temperature
  rather than being selected randomly as is the established case.


\end{abstract}

\begin{IEEEkeywords}
Simulated annealing, partitioning, heterogeneous architectures.

\end{IEEEkeywords}


% For peer review papers, you can put extra information on the cover
% page as needed:
% \ifCLASSOPTIONpeerreview
% \begin{center} \bfseries EDICS Category: 3-BBND \end{center}
% \fi
%
% For peerreview papers, this IEEEtran command inserts a page break and
% creates the second title. It will be ignored for other modes.
\IEEEpeerreviewmaketitle



\section{Introduction}

% Forthcoming systems are going to be \textit{Multi-Processor System on
%   Chip} (MpSoC) consisting of a heterogeneous mix of cores such as CPUs,
% GPUs, hardware acceleration units, etc. Examples of such systems are:
% ARM Cortex-A9, AMD Zacate, and STMicroelectronics P2012
% platforms~\cite{cortex9a,zacate,lben10}. 

% On the one hand hardware industry is overcoming Moore's law by moving to
% very large scale multi-processor systems, on the other hand software
% design is playing catchup. Programming simple multi-core systems using
% the traditional threading model is considered hard~\cite{elee06}. 

% The exponentially growing number of heterogeneous multi-core systems, in
% hand-held devices, desktops, server farms, etc, exacerbates the problem
% of extracting parallelism from such hardware, exponentially. A number of
% different programming paradigms have been proposed to overcome this lack
% in designer productivity. Two of the most promising techniques are:
% stream based programming~\cite{elee871}, which generalizes the Kahn
% process networks~\cite{gkan74}. CUDA~\cite{jsan10} and
% OpenCL~\cite{opencl08}, both targeted at GPUs follow the streaming model
% of computation. The other popular approach to programming large
% multi-core systems is mixing OpenMp~\cite{qmic04} with vectorization
% techniques automatically extracted from the traditional C programs using
% techniques such as the Polytope model~\cite{mgri98}. The second
% alternative is used in GCC and LLVM compiler infrastructures.

% While the streaming approach provides programming level tools (usually
% in the form of pragmas or new programming languages) for the software
% designer to exploit parallelism. The compiler approach tries to
% automatically extract parallelism from the given application. Both these
% approaches have advantages and disadvantages. But, they both share a few
% points in common: (1) The streaming and the OpenMp approaches both
% exhibit task parallelism. Task-parallelism is the form of parallelism
% most commonly exploited by programmers on \textit{Multiple Instruction
%   Multiple Data} (MIMD) architectures. (2) Streaming like vectorization
% also exhibits data-parallelism, which is most suited for \textit{Single
%   Instruction Multiple data} (SIMD) architectures such as GPUs or vector
% units on a CPU. (3) Multi-core processors exhibit the phenomenon of
% \textit{Non Uniform Memory Access} (NUMA), which needs to be considered
% when exploiting the aforementioned forms of parallelism. NUMA
% essentially requires the designers and compiler writers to consider the
% placement of data when partitioning the application. The data-placement
% problem is further exacerbated with the problem of caches, and latencies
% of memory transfers between differing processing elements, especially
% hosts and devices connecting CPUs and GPUs.

There are a plethora of different optimization techniques that have been
proposed for extracting data-parallelism from
loops~\cite{smuc97,mgri98}. In case of task-parallelism the automatic
extraction of parallelism has been recently proposed using OpenMP using
the Polytope model~\cite{mgri98}, while the streaming community
(repsenting CUDA~\cite{jsan10} and OpenCL~\cite{opencl08}) has its own
optimization techniques~\cite{mgor06}. Moreover, the design of the
application heavily influences the potential for parallelism
extraction. 


\begin{figure}[h!]
  \centering
\begin{verbatim}
 for (int i=0;i<M; ++i){
  for (int j=0;j<N; ++j){
   1: A[i][j] = (i*j+4.0/N)
   2: B[i][j] = (i*j+9.0/N)
  }
 }
\end{verbatim}
  \caption{Example C source code, filling in two matrices in a stencil
    computation}
  \label{fig:1}
\end{figure}

Consider the simple code snippet shown in Figure~\ref{fig:1}. This code
snippet can be re-written to extract task-parallelism as shown in
Figure~\ref{fig:2}. The two nested for loops (once fissed) can now be
carried out on two different processor cores. We define task-parallel
processes as either nodes that consist of complex computations, but are
independent of each other. Or as loops that consists of a series of
statements that are independent.

\begin{figure}[h!]
  \centering
\begin{verbatim}
for (int i=0;i<M;++i)
 for (int j=0;j<N;++j)
  A[i][j] = i*j+4.0/N

for (int i=0;i<M;++i)
 for (int j=0;j<N;++j)
  B[i][j] = i*j+9.0/N
\end{verbatim}
  
  \caption{Fissed loops to extract task-parallelism}
  \label{fig:2}
\end{figure}

Similarly, one can also extract data-parallelism by further modifying
Figure~\ref{fig:2}, as shown in Figure~\ref{fig:3}. As seen in
Figure~\ref{fig:3}, the assignment to \texttt{A} and \texttt{B} are
carried out in parallel using different cores, but, while the assignment
to \texttt{A} is carried out in a vector instruction for the just the
inner loop, for \texttt{B}, both the loops are collapsed, and the
vectorization is carried out in a single go. The two options chosen
depend upon the underlying architecture. The size of the caches and the
size of underlying vector length. Optimizations can be performed further
still on Figure~\ref{fig:3}, such as loop unrolling or loop interchange.

\begin{figure}[h!]
  \centering
\begin{verbatim}
for (int i=0;i<M;++i)
//inner loop vectorization
  A[i][0:N-1] = i*[0:N-1]+4.0/N

// Loop collapsed vectorization
  B[0:M-1][0:N-1] = [0:M-1]*[0:N-1]+9.0/N
\end{verbatim}

  \caption{Data and task parallel for loops}
  \label{fig:3}
\end{figure}

Different parts of the applications can exhibit differing task and
data-parallel potentials. Furthermore, the amount of communication
between these differing parts might vary itself. Given a heterogeneous
architecture consisting of processing elements with varying vector
lengths, varying processing capacity and connected via differing
bandwidths/latencies, how does one compile or even design the program to
actually utilize the underlying hardware properly? This is the question
we are trying to answer in this paper. We provide an improved simulated
annealing approach to solving this problem. Our key contributions in
this paper are:

\begin{itemize}
\item A Simulated annealing approach to tuning programs for task and
  data parallelism across heterogeneous multi-core architectures.
\item An improvement in the simulated annealing algorithm, which
  empirically shows that the resulting partitions have smaller execution
  times.
\end{itemize}


\section{Related Work}
\label{sec:related-work}

There has been tremendous advances in compiler design for loop
optimizations~\cite{ubon08,atiw09,tkis00}. The tiles of loop sizes, the
unroll factor, etc, have been studied using a plethora of heuristic
techniques, including simulated annealing. Tiwari et.al.~\cite{atiw09},
describe an auto-tuning compiler, which iteratively solves the loop
tiling problem. They along with others~\cite{tkis00} have also looked at
loop unrolling and blocking and jamming techniques to extract vectorized
parallelism. Overall, we have found their techniques to be quite
effective in real-world applications. What, this optimization literature
does not address is the problem of compiler optimizations for a
heterogeneous processing architecture. Furthermore, the researchers have
also not addressed the problem of extracting a combination of task and
data-parallelism as described in the previous section. 

On the other hand we have the streaming and task distribution community,
who have started probing the problem of partitioning and task
distribution~\cite{ssan05,adou04,pcar09} onto heterogeneous processing
architecture. The streaming community especially recognizes the need to
extract both task and data-parallelism (which corresponds to loop level
parallelism)~\cite{mgor06}. Knowing that the task partitioning problem
is known NP-hard, the researchers take a heuristic approach.  But, yet
again, the streaming community seems to not make a connection between
the polyhedral compiler optimization community and itself. In this paper
we bridge the gap between the two optimization techniques, thereby
leading to a powerful optimization framework, which can be used by
compiler writers or application designers to explore a heterogeneous
search space.

% It is expected that these MpSoCs would play a significant role in the
% hand held device industry, mostly for the (soft) real-time applications
% such as voice recognition, video and audio capture, decoding, and
% playback. Software compilation techniques for such complex MPSoCs is
% largely lagging behind the developments in hardware. Application
% developers still distribute and schedule threads onto these multi-core
% systems by hand, which is a time consuming process and leads to low
% designer productivity. It is expected that the research community would
% play an important role in providing the state of the art
% \textit{automated} compilation support for these large heterogeneous
% MPSoCs in order to increase the designer productivity and fasten the
% time to market.

% A number of applications running on these hand-held devices exhibit
% heterogeneity themselves. We define heterogeneity in the applications as
% differing computation and communication requirements. Consider the a 

% The aforementioned signal processing applications can be characterized
% using the \textit{Static Data Flow} (SDF) model~\cite{wthi02}, where
% data flows from one filter to the next. The streaming model has been
% shown to be a nice fit for \textit{Single Instruction Multiple Data}
% (SIMD), \textit{Multiple Instruction Multiple Data} (MIMD)
% architectures~\cite{iban04,cilk_plus_plus,mgor06}. This resurgence in
% the SIMD and MIMD processors (GPUs, IBM CELL, etc) has rekindled the
% scientific community, which in turn has led to significant research
% interest in compilation of SDF graphs onto these
% architectures~\cite{fsar11,mkud08,mgor06,audu09,pcar09}.

% Another very important aspect that is unanswered in the current
% literature is scheduling models for cyclic SDF graphs. Although one
% might argue that cycles do not occur often in stream graphs, it is
% obvious that most \textit{smart} applications need some kind of feedback
% control, which translates to cycles within the SDF graphs. Current
% attempts at incorporating software pipelined execution of SDF graphs,
% e.g.,~\cite{fsar11}, model software pipelined execution as a bin-packing
% problem. Bin-packing is not the correct model for cyclic SDF graphs (as
% will be shown in Section~\ref{sec:part-sdf-graphs}). Other
% attempts~\cite{rgov02,audu09} at incorporating software pipelined
% execution of cyclic SDF graphs either ignore cycles encapsulating
% branches or provide a sub-optimal solution.

% Finally, stateless actors (e.g., \texttt{FFT}) occur often in stream
% graphs. Extracting parallelism from these stateless actors is
% non-trivial. A number of research directions have been proposed that can
% utilize the parallelism hidden inside stateless
% actors~\cite{fsar11,mkud08,mgor06}. But, none of these approaches
% provide an optimal solution to this problem, i.e., none of the
% approaches is able to determine the optimal number of copies that need
% to be created and run in parallel given a heterogeneous processor
% architecture. We present an solution to this problem.

\section{Notations and formalization of the problem statement}
\label{sec:relat-notat-form}

The overall application partitioning problem onto a heterogeneous
execution architecture can formulated in terms of a graph partitioning
problem. Given two graphs, the task-graph representing the application
and the resource-graph representing the underlying topology. How does
one map the task-graph onto the resource-graph.


\subsection{Formalizing the task-graph}
\label{sec:form-task-graph}

A task-graph is a graph weighted directed graph $G_t(V_t,E_t)$, such
that each vertex $V_t$ is an execution node in the application, and
\mbox{$E_t \subseteq (V_t \times V_t)$}, shows the communication edges
between the vertices. Each vertex $V_t$ is decorated with weights:
$w^t_0(i), \forall i \in V_t$ and $w^t_1(i), \forall i \in V_t$. Weight
$w^t_0$ is a function on some vertex $i \in V_t$, that maps the number
of instructions to be carried out at node $i$ to an integer
value. Similarly, $w^t_1$ is a function on some vertex $i \in V_t$ that
maps the vector length that is required by the node $i$ again to an
integer value. Each edge is decorated by a weight $w^e(e), \forall e \in
(V_t \times V_t)$. Weight $w^e$ represents the number of bytes that need
to be transfered from the location of the data-store ($i$) to the
utilization node ($j$).


\subsection{Formalizing the resource graph}
\label{sec:form-reso-graph}

The system resources are represented by a weighted undirected graph
$G_r(V_r,C_r)$. Where $V_r$ represents a processing element in the
underlying resource graph, while the edge $C_r \subseteq (V_r \times
V_r)$, represents the communication links. Each vertex is decorated with
a weights $W^r_0(i)$ and $W^r_1(i), \forall i \in V_r$, which map the
\textit{Million Instructions Per Second} (MIPS) count of the vertex to
an integer value, and the vector capacity of that vertex again to an
integer domain, respectively. Every, communication link is weighted with
the bandwidth capacity denoted by $W^c(c), \forall c \in (V_r \times
V_r)$.

\subsection{Formalizing the objective: the latency of the application}
\label{sec:form-latency-appl}

The total application latency in a parallel setting is the addition of
the computation time and the communication time. Given some application
node $i \in V_t$ mapped to some resource $j \in V_r$. The latency for
that node is computed as: $((w^t_1(i)/W^r_1(j)\times w^t_0(i))/W^r_0(c))
+ (w^e(e)/W^c(c)) | e = (i,k), k \neq i, \forall k \in V_t, c = (j,l), l
\neq j, \forall l \in V_r $. In this formulation for some given
task-graph node $i$, we first calculate the number of vectorized
instructions that need to be performed (by diving the required vector
length with the vector capacity of the resource node), this gives us the
total number of vector instructions that would be performed on the
resource node $j$. Next, we multiply the number of vector instructions
to be performed by the number of iterative (non-vectorized) instructions
required, this in turn gives us the total number of instructions to be
performed by that task-graph node. Finally, we find the computation
latency by divining this total number of instructions with the MIPS
value of the resource vertex. The communication on the other hand is a
quite simple, we calculate the communication latency, by dividing the
number of required bits to be transfered by the bandwidth of the
resource.

Given the task-graph and the resource-graph, let $\zeta$ be all possible
mappings of the application on the resource-graph. For a particular
mapping $\mathcal{M}$ defined as $\zeta_\mathcal{M}$ on some resource $s
\in V_r$ the mapping latency is defined as:

\begin{equation}
  \begin{array}{c}
    Latency^{\zeta_\mathcal{M}}_s = \\
    \\
    \sum_{\forall i \in V_t \wedge
      \zeta_\mathcal{M} = s} ((w^t_1(i)/W^r_1(s)\times w^t_0(i))/W^r_0(c))
    \\
    +
    \\
    \sum_{\forall i \in V_t \wedge
      \zeta_\mathcal{M} = s} w^e(e) / W^c(c)\\ 
    s.t., e = (i,k), k \neq i, \forall k
    \in V_t \wedge\  c = (s,l), l \neq s, \forall l \in V_r
  \end{array}
  \label{eq:1}
\end{equation}

Finally, the complete application latency can then be defined as: 
\begin{equation}
  \label{eq:2}
  Latency^{\zeta_\mathcal{M}} = max_{s}
  ({Latency^{\zeta_\mathcal{M}}_s}), \forall s \in V_t
\end{equation}

The objective of our framework is to minimize the total application
latency as described in Equation~(\ref{eq:2}).

%%% Local Variables: 
%%% mode: latex
%%% TeX-master: "bare_conf"
%%% End: 


\input{task_graph}

\section{Simulated annealing}
\label{sec:simulated-annealing}

Simulated Annealing~\cite{kirkpatrick} is a generic probabilistic algorithm for
finding the global optimum of a given function,
$\mathcal{F}:\mathcal{R}\rightarrow\mathcal{R}$, which has a large search space.
Simulated Annealing in most of the cases is cheaper than exhaustive enumeration,
which is exhaustively listing the entire search space. Since it is a monte carlo
method it will always terminate and the solution need not be the most optimal
but in general considered to be a good approximation of the optimum.

\subsection{Overview}

In general SA is a non-greedy heuristic algorithm which explores the search
space by moving from a higher temperature to a lower temperature. The algorithm
always accepts moves to a better solution, determined by evaluating an
\emph{objective function} and sometimes accpets moves to worse states with a
probability that changes with temperatures. Initially, when the temperature is
high, the algorithm is more tolerable and accepts moves to worse states with a
higher probability. As the temperature falls down, this probability decreases
and the alogrithm is less tolerant towards solutions with a worse value for the
objective function.

This is the reason why SA is not a purely greedy algorithm as it accepts moves
to bad states with some probability which enables it to move out of local
maximas\/minimas. But the algorithm becomes more greedier as the temperature
decreases. Figure~\ref{fig:temperature_drop} shows how the temperature drops and
how this acceptance probability changes with this drop in temperature.

\subsection{From the mapping problem standpoint}

\begin{algorithm}
\DontPrintSemicolon
\KwIn{Initial Mapping $\zeta_0$ and Starting Temperature $T_0$}
\KwOut{Best Mapping $\zeta_{best}$}
\begin{algorithmic}[1]
$\zeta \leftarrow \zeta_0$ \\
$C \leftarrow OBJECTIVE\_FUNCTION(\zeta_0)$ \\
$\zeta_{best} \leftarrow \zeta$
$C_{best} \leftarrow C$

\caption{$Simulated\_Annealing(\zeta_0, T_0)$}
\label{algo:SA}
\end{algorithmic}
\end{algorithm}

Figure~\ref{algo:SA} represents the pseudocode for our algorithm. We begin by
choosing some initial starting point for our state space exploration, which from
the mapping problem standpoint is some initial mapping of tasks from $V_t$ to
PEs $V_r$. We choose our initial mapping to be all tasks from $V_t$ mapped onto
some random PE $j \in V_r$. This is a good idea because we have observed that
this puts tightly coupled tasks, i.e. tasks that communicate heavily, on the
same PE since moving them onto different PEs would incur more costs and would thus
lead to a bad state. 

\subsection{Parameter Selection}


%%% Local Variables: 
%%% mode: latex
%%% TeX-master: "bare_conf"
%%% End: 


\section{Experiments and Results}
We show the speedup obtained using our improved heuristics against the conventional
heuristics for Simulated Annealing as prescribed by~\cite{heikki}. We also
compare the results we obtain against the heterogeneous bin packing algorithm as
explained below. 

\subsection{Heterogeneous Bin Packing}

To determine how good our framework we compare it against an adaptation of the
well known heterogeneous bin packing solutions as discussed by Teodor et
al.~\cite{CIRRLET}. They adapt the well-known \textit{Best First Decreasing
(BFD)} heuristic, which works only for homogeneous bins, to form the
\textit{Adapted BFD (A-BFD)} heuristic which works for heterogeneous bins. A
small gist of the algorithm is as follows.

Let $\mathcal{I}$ be the items to be accommodated into the bins and let $\mathcal{K}$ be the set of bins available.
From the standpoint of the mapping problem, $\mathcal{I}$ refers to the set of
application-tasks($V_t$) and $\mathcal{K}$ refers to the resources($V_r$). Similar to
the Knapsack problem~\cite{knapsack}, by which A-BFD is inspired, each
element $i \in \mathcal{I}, \mathcal{K}$ has two constraints on them represented
by $c_i$(cost) and $V_i$(volume). Right off the bat, this seems to be a significant 
advantage, our framework has over heterogeneous bin packing, as \textit{A-BFD} only
works with two constraints whereas our framework poses no such restrictions.

\textit{A-BFD} proceeds to sort $\mathcal{I}$ according to non-increasing order
of their volume and sorts $\mathcal{K}$ according to non-increasing order of the
ratio $c_i/V_i$. Then, it proceeds to allocate items from $\mathcal{I}$ into
best bins $b \in \mathcal{S}$. A ``best" bin, i.e., the bin with maximum free
space, is defined as the bin volume minus the sum of volumes of the items loaded
into it. In the mapping problem setting, this becomes a double edged sword as it
boils down to a single constraint solving problem giving higher priority to
the second constraint (vector requirement of the application-tasks $w^t_1(i)$ and
vector capability of the resources $W^r_1(i)$). 

The post pass in \textit{A-BFD} chooses every bin that has atleast one item
allocated to it and tries to find an empty bin, that has a higher or equal volume
than the allocated volume on the chosen bin but also has a lower cost. If it
finds such an empty bin, then it transfers all the items allocated to the chosen
bin to the newly found empty bin which is cheaper. One of the main advantages of
\textit{A-BFD} is that it is very fast with a best case complexity of
$O(N_\mathcal{I})$ without the post pass, where $N_\mathcal{I}$ is the number of
items(number of tasks $N_T$ in the application graph $V_t$). Including the post
pass, the best case complexity becomes $O(N_\mathcal{I} + N_\mathcal{K})$ where
$N_\mathcal{K}$ is the number of bins(number of PEs $N_R$ in the resource graph
$V_r$)

\section{Conclusion}


% \section*{Acknowledgment}


% The authors would like to thank...
% more thanks here


\scriptsize{
\bibliographystyle{IEEEtran}
% \bibliography{latex8}
\bibliography{/Users/amal029/Dropbox/BIBLIOGRAPHY_DATABASE/main_bib.bib}
}

% that's all folks
\end{document}



%%% Local Variables: 
%%% mode: latex
%%% TeX-master: t
%%% End: 
